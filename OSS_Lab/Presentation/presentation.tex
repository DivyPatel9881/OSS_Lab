\documentclass{beamer}
\usetheme[faculty=fi]{fibeamer}
\usepackage[utf8]{inputenc}
\usepackage[
  main=english, %% By using `czech` or `slovak` as the main locale
                %% instead of `english`, you can typeset the
                %% presentation in either Czech or Slovak,
                %% respectively.
  czech, slovak %% The additional keys allow foreign texts to be
]{babel}   

\title{
 PYTHON\\ DIVY PATEL 171080056\\ Under the Guidance of Pranav Sir} %% that will be typeset on the
  \subtitle{Quick Overview} %% title page.
  \author{Divy Patel}
\begin{document}
  \frame{\maketitle}
 \begin{darkframes}
    \begin{frame}{CONTENTS}
    \tableofcontents
    \section{HISTORY}
    \section{FEATURES}
    \section{INBUILT OBJECTS}
    \section{OPERATORS}
    \section{FUNCTIONS}
    \section{CONTROL STRUCTURES}
    \section{REFERENCES}
    \end{frame}
    
     \begin{frame}{ \textbf{HISTORY}}
        \begin{itemize}
         \item Python was conceived in the late 1980s by Guido van Rossum at Centrum Wiskunde & Informatica (CWI) in the Netherlands as a successor to the ABC language (itself inspired by SETL), capable of exception handling and interfacing with the Amoeba operating system. 
         \item Its implementation began in December 1989. Van Rossum's long influence on Python is reflected in the title given to him by the Python community: Benevolent Dictator For Life (BDFL) – a post from which he gave himself permanent vacation on July 12, 2018.
         \item Python 2.0 was released on 16 October 2000 with many major new features, including a cycle-detecting garbage collector and support for Unicode.
        \end{itemize}
    \end{frame}
    
    \begin{frame}{\textbf{FEATURES}}
    \begin{itemize}
        \item Python is a multi-paradigm programming language. Object-oriented programming and structured programming are fully supported, and many of its features support functional programming and aspect-oriented programming (including by metaprogramming and metaobjects (magic methods)).
        \item Many other paradigms are supported via extensions, including design by contract and logic programming.
        \item Python uses dynamic typing, and a combination of reference counting and a cycle-detecting garbage collector for memory management. \item It also features dynamic name resolution (late binding), which binds method and variable names during program execution.
    \end{itemize}
    \end{frame}
    
    \begin{frame}{\textbf{INBUILT OBJECTS}}
    \begin{tabular}{|c|c|c|}
        \hline
        \textbf{Types} & \textbf{Mutability} & \textbf{Description}\\
        \hline
         bytes & Immutable & Sequence of bytes\\
         \hline
         str & Immutable & A Sequence of unicode codepoints \\
         \hline
         int & Immutable & Integer of unlimited magnitude\\
         \hline
         float & Immutable & Floating Point Number\\
         \hline
         bool & Immutable & Boolean value\\
         \hline
         list & Mutable & List, can contain mixed types\\
         \hline
         set & Mutable& Unordered set, contains no duplicates\\
         \hline
         dict & Mutable& Associative arrays, with key-value pairs\\
         \hline
    \end{tabular}
    
    \end{frame}
    
    \begin{frame}{\textbf{OPERATORS}}
    \begin{tabular}{|c|c|}
    \hline 
    \textbf{Operator} & \textbf{Description} \\
    \hline
    + Addition & Adds values on either side of operator.\\
    \hline
    - Substraction & Substracts right operand from left operand.\\
    \hline 
    * Multiplication & Multiply operands on either side.\\
    \hline 
    / Division & Divide right operand from the left operand.\\
    \hline 
    \% Modulus & Gives Remainder.\\
    \hline 
    ** Division & Power of left raised by right operand.\\
    \hline
    \end{tabular}
    \end{frame}
    
    \begin{frame}{\textbf{DEFINING THE FUNCTIONS}}
    You can define functions to provide the required functionality. Here are simple rules to define a function in Python.
    \begin{itemize}
        \item Function blocks begin with the keyword def followed by the function name and parentheses ( ( ) ).
        \item Any input parameters or arguments should be placed within these parentheses. You can also define parameters inside these parentheses.
        \item The code block within every function starts with a colon (:) and is indented.
        \item The statement return [expression] exits a function, optionally passing back an expression to the caller. A return statement with no arguments is the same as return None.
    \end{itemize}
    \end{frame}
    
    \begin{frame}{\textbf{FUNCTIONS SYNTAX}}
        \begin{block}{Function Declaration}
      def FunctionName (params):\\
        \\\quad  \# body of the function
        \\ \quad Statements we want to execute\\
    \end{block}
    \begin{block}{Function Calling}
      FunctionName (args)\\
    \end{block}
    \end{frame}
    
  \begin{frame}{\textbf{Control Structure}}
    \begin{block}{FOR LOOP}
      for i in iterable :\\
        \\\quad  \# body of the loop
        \\ \quad Statements we want to execute\\
    \end{block}
    
    \begin{block}{WHILE LOOP}
      while condition :\\
        \\\quad  \# body of the loop
        \\ \quad Statements we want to execute\\
     \end{block}
    
    \begin{block}{IF STATEMENT}
      if condition :\\
        \\\quad  \# body 
        \\ \quad Statements we want to execute\\
     \end{block}
     
  \end{frame}
  
  \begin{frame}{\textbf{CONTROL STRUCTURES}}
     \begin{block}{IF-ELSE STATEMENT}
      if condition :\\
        \\\quad  \# body 
        \\ \quad Statements we want to execute\\
      else : \\
        \\\quad  \# body 
        \\ \quad Statements we want to execute\\
     \end{block}
  \end{frame}
  
  \begin{frame}{\textbf{CONTROL STRUCTURES}}
     \begin{block}{ELSE-IF STATEMENT}
      if condition1 :\\
        \\\quad  \# body 
        \\ \quad Statements we want to execute\\
      elif condition2 : \\
        \\\quad  \# body 
        \\ \quad Statements we want to execute\\
     \end{block}
  \end{frame}
 
  \begin{frame}{\textbf{REFERENCES}}
\begin{thebibliography}{9}
\bibitem{python}
Python Official Docs \\
\texttt{https://www.python.org/doc/}
 
\bibitem{tutorialspoint}
Tutorials Point \\
\texttt{https://www.tutorialspoint.com/python/}

\bibitem{geeksforgeeks}
GeeksForGeeks
\\\texttt{https://www.geeksforgeeks.org/python-programming-language/}
\end{thebibliography}
  \end{frame}
    \end{darkframes}
\end{document}